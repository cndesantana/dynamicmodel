% Template for PLoS
% Version 1.0 January 2009
%
% To compile to pdf, run:
% latex plos.template
% bibtex plos.template
% latex plos.template
% latex plos.template
% dvipdf plos.template

\documentclass[10pt]{article}

% amsmath package, useful for mathematical formulas
\usepackage{amsmath}
% amssymb package, useful for mathematical symbols
\usepackage{amssymb}

% graphicx package, useful for including eps and pdf graphics
% include graphics with the command \includegraphics
\usepackage{graphicx}
\usepackage{times}
\usepackage{tikz}
\usepackage{amsmath}
\usepackage{verbatim}
\usepackage{animate}
\usetikzlibrary{arrows,shapes}

% cite package, to clean up citations in the main text. Do not remove.
\usepackage{cite}

\usepackage{color} 

% Use doublespacing - comment out for single spacing
%\usepackage{setspace} 
%\doublespacing


% Text layout
\topmargin 0.0cm
\oddsidemargin 0.5cm
\evensidemargin 0.5cm
\textwidth 16cm 
\textheight 21cm

% Bold the 'Figure #' in the caption and separate it with a period
% Captions will be left justified
\usepackage[labelfont=bf,labelsep=period,justification=raggedright]{caption}


% Use the PLoS provided bibtex style
\bibliographystyle{plos2009}

% Remove brackets from numbering in List of References
\makeatletter
\renewcommand{\@biblabel}[1]{\quad#1.}
\makeatother


% Leave date blank
\date{}

\pagestyle{myheadings}
%% ** EDIT HERE **


%% ** EDIT HERE **
%% PLEASE INCLUDE ALL MACROS BELOW

%% END MACROS SECTION

\begin{document}

% Title must be 150 characters or less
\begin{flushleft}
{\Large
\textbf{Self-organized predation and migration model}
}
%% Insert Author names, affiliations and corresponding author email.
%\\
%Charles Novaes de Santana$^{1,2,\ast}$ 
%\\
%\bf{1} Author1 Dept/Program/Center, Institution Name, City, State, Country
%\\
%\bf{2} Author2 Dept/Program/Center, Institution Name, City, State, Country
%\\
%\bf{3} Author3 Dept/Program/Center, Institution Name, City, State, Country
%\\
%\bf{4} Author3 Dept/Program/Center, Institution Name, City, State, Country
%\\
%\bf{5} Author3 Dept/Program/Center, Institution Name, City, State, Country
%\\
%$\ast$ E-mail: charles.desantana@eawag.ch 
\end{flushleft}

% Please keep the abstract between 250 and 300 words
\section*{Abstract}

\cite{desantana2013topological}
% Please keep the Author Summary between 150 and 200 words
% Use first person. PLoS ONE authors please skip this step. 
% Author Summary not valid for PLoS ONE submissions.   
\section*{Author Summary}

\section*{Introduction}

% Results and Discussion can be combined.
\section*{Results}

\vspace{0.25cm}
\subsection*{Steady State}

\vspace{0.25cm}
\subsection*{Space of Parameters}

\vspace{0.25cm}
\subsection*{Metabolic Theory of Ecology}

\section*{Discussion}

% You may title this section "Methods" or "Models". 
% "Models" is not a valid title for PLoS ONE authors. However, PLoS ONE
% authors may use "Analysis" 
\section*{Materials and Methods}

\vspace{0.25cm}
\subsection*{Fluxogram of the model}

Please see Figure \ref{fig:Fluxogram}

\vspace{0.25cm}
\subsection*{Predation Equations}

\vspace{0.25cm}
\begin{itemize}
\item \textbf{General Equation}
\end{itemize}

\begin{align*}
 N^\prime\left(sp\right) & = N\left(sp\right) \times \left\{ \tikz[baseline]{\node[fill=blue!20,anchor=base] (t1) { $\left[ 1 - NDp\left(sp\right) \right]\left[ \sum_{b \in H\left(sp\right)}\rho\left(b\right)Dp\left(b\right) \right]\left[Bp\left(sp\right)\right]$ };}  \right.\nonumber  \\
 &\qquad \left. \tikz[baseline]{\node[fill=red!20,anchor=base] (t2) {$- \left[\sum_{c \in P\left(sp\right)}\rho\left(c\right)\left( 1-NDp \left(c \right) \right)\frac{\rho\left(sp\right)}{\sum_{d \in H\left(c\right)}\rho\left(d\right)}Dp\left(sp\right)\right]-\left[NDp\left(sp\right)\right]$ };} \right\} 
\end{align*}

%%%%%%%%%%%%%%%%%%
%%%%%%%%%%%%%%%%%%% B I R T H      P R O B A B I L I T Y
%%%%%%%%%%%%%%%%%%
\vspace{1cm}
\begin{itemize}
\item \textbf{Self-Organized Parameters: Birth Probability}
\end{itemize}

\begin{align*}
Bp\left(sp\right) & = \tikz[baseline]{\node[fill=blue!20,ellipse,anchor=base] (t1) {$\left[1 - \rho\left(sp\right)\right]$};} \times \tikz[baseline]{\node[fill=red!20,anchor=base] (t2) {$\left[\sum_{b \in H\left(sp\right)}\rho\left(b\right)\left(1-\sum_{c \in P\left(b\right)}\rho\left(c\right)\right)\right]$};} \\
& \times \tikz[baseline]{\node[fill=green!20,anchor=base] (t3) {$\left[1-\sum_{c \in P\left(sp\right)}\rho\left(c\right) \right]$};}
\end{align*}

\textbf{Where:} Availability of Resources of Basal Species is 1.0

%%%%%%%%%%%%%%%%%%%%%%%%%%
%%%%%%%%%%%%%%%%%%%%%%%%%%% D E A T H     P R O B A B I L I T Y
%%%%%%%%%%%%%%%%%%%%%%%%%%
\vspace{1cm}
\begin{itemize}
\item \textbf{Self-Organized Parameters: Death Probability}
\end{itemize}

\begin{align*}
Dp\left(sp\right) & = \tikz[baseline]{\node[fill=blue!20,ellipse,anchor=base] (t1) {$\left[\rho\left(sp\right)\right]$};} \times \tikz[baseline]{\node[fill=red!20,anchor=base] (t2) {$\left[\sum_{b \in H\left(sp\right)}\left(1-\rho\left(b\right)\right)\left(\sum_{c \in P\left(b\right)}\rho\left(c\right)\right)\right]$};} \\ 
& \times \tikz[baseline]{\node[fill=green!20,anchor=base] (t3) {$\left[1-\sum_{c \in P\left(sp\right)}\rho\left(c\right) \right]$};} 
\end{align*}

\textbf{Where:} Death Probability of Basal Species is 1.0

%%%%%%%%%%%%%%%%%%%%%%%%%%
%%%%%%%%%%%%%%%%%%%%%%%%%%% N A T U R A L     D E A T H     P R O B A B I L I T Y
%%%%%%%%%%%%%%%%%%%%%%%%%%
\vspace{1cm}
\begin{itemize}
\item \textbf{Self-Organized Parameters: Natural Death Probability}
\end{itemize}

\begin{align*}
NDp\left(sp\right) & = \tikz[baseline]{\node[fill=blue!20,ellipse,anchor=base] (t1) {$\left[\rho\left(sp\right)\right]$};} \times \tikz[baseline]{\node[fill=red!20,anchor=base] (t2) {$\left[\sum_{b \in H\left(sp\right)}\left(1-\rho\left(b\right)\right)\left(\sum_{c \in P\left(b\right)}\rho\left(c\right)\right)\right]$};} \\ 
& \times \tikz[baseline]{\node[fill=green!20,anchor=base] (t3) {$\left[1-\sum_{c \in P\left(sp\right)}\rho\left(c\right) \right]$};} 
\end{align*}

%%%%%%%%%%%%%%%%
%%%%%%%%%%%%%%%% S P E C I E S     C A R R Y I N G     C A P A C I T Y
%%%%%%%%%%%%%%%%
\vspace{1cm}
\begin{itemize}
\item \textbf{Self-Organized Parameters: Carrying Capacity (for each species in each site)}
\end{itemize}

\begin{align*}
CC\left(sp\right) & =  \tikz[baseline]{\node[fill=red!20,ellipse,anchor=base] (t2) {$\left[\sum_{b \in H\left(sp\right)}\frac{\left(\rho\left(b\right)\right)}{\left(\sum_{c \in P\left(b\right)}\rho\left(c\right)\right)}\right]$};}
\end{align*}

%%%%%%%%%%%%%%%%
%%%%%%%%%%%%%%%% M O B I L I T Y 
%%%%%%%%%%%%%%%% 

\vspace{0.25cm}
\subsection*{Migration Equations}

%%%%%%\begin{frame}
%%%%%%\frametitle{Mobility Dynamics}
%%%%%%%\includegraphics[width=1.0\textwidth]{./mobility.eps}
%%%%%%\end{frame}
%%%%%%

\vspace{1cm}
\begin{itemize}
    \item \textbf{Mobility of species \emph{sp} from site \emph{i} to \emph{j}} 
\end{itemize}

$\Delta N_{sp}\left(i\right) = \sum_{j \in Neigh\left(i\right)} \left( \left( N_{sp}(j) \; M_{sp}(j,i) \; - \; \tikz[baseline]{\node[fill=blue!20,anchor=base] (t1) {$N_{sp}(i) $};} \tikz[baseline]{\node[fill=red!20,anchor=base] (t2) {$M_{sp}(i,j)$};} \right) \right)$


\vspace{0.25cm}
\small{$M(i,j) = \overbrace{ \left[ \lambda^{i} \frac{ \Delta_{ij}\,f\,\Theta \left(\Delta_{ij}\,f\,\right) }{ \sum_{k \in Neigh\left(i\right)}\Delta_{ik}\,f\,\Theta \left(\Delta_{ik}\,f\,\right) }\right]}^{Biotic} \overbrace{\left[w_{ij} \frac{ \Delta_{ij}\,f_{\eta}\,\Theta \left(\Delta_{ij}\,f_{\eta}\,\right) }{ \sum_{k \in Neigh\left(i\right)}\Delta_{ik}\,f_{\eta}\,\Theta \left(\Delta_{ik}\,f_{\eta}\,\right) }\right]}^{Abiotic}$}

\vspace{0.5cm}
$$
\Delta_{ij}\,f = f^{i,j} - f^{j,i}\left\{ \begin{array}{rl}
 f^{i,j} = \rho_{H}(j) + \rho_{P}(i) \\
 f^{j,i} = \rho_{H}(i) + \rho_{P}(j) \\
       \end{array} \right.
$$

\vspace{0.5cm}
\centering $\lambda^{i} = \frac{1}{2} \left( 1 - RE^{i} \right)$ \\

\vspace{0.25cm}
\centering $ RE^{i} = \frac{ New^{t} }{ N^{t} }$

\vspace{0.25cm}
$$
 \Delta_{ij}\,f_{\eta} = f_{\eta}^{j} - f_{\eta}^{i} \left\{ \begin{array}{rl}
 f_{\eta}^{i} = \eta_{sp}^{*} - \eta_{sp}^{i} \\
 f_{\eta}^{j} = \eta_{sp}^{*} - \eta_{sp}^{i} \\
       \end{array} \right.
$$

\vspace{0.25cm}
$$w_{ij} = Connectivity \: between \: sites \: i and \: j$$

%%%%%%%%%%%%%%%%%%%%%% 
%%%%%%%%%%%%%%%%%%%%%%

\vspace{1cm}
% Do NOT remove this, even if you are not including acknowledgments
\section*{Acknowledgments}


%\section*{References}
% The bibtex filename
\bibliography{references}

\vspace{1cm}
\section*{Figures}

\begin{figure}[!ht]
\begin{center}
\includegraphics[width=4in]{./figures/Fluxogram.eps}
\end{center}
\caption{
{\bf Fluxogram.} Fluxogram of the model.
}
\label{fig:Fluxogram}
\end{figure}

%\begin{figure}[!ht]
%\begin{center}
%%\includegraphics[width=4in]{figure_name.2.eps}
%\end{center}
%\caption{
%{\bf Bold the first sentence.}  Rest of figure 2  caption.  Caption 
%should be left justified, as specified by the options to the caption 
%package.
%}
%\label{Figure_label}
%\end{figure}

\vspace{1cm}
\section*{Tables}
%\begin{table}[!ht]
%\caption{
%\bf{Table title}}
%\begin{tabular}{|c|c|c|}
%table information
%\end{tabular}
%\begin{flushleft}Table caption
%\end{flushleft}
%\label{tab:label}
% \end{table}

\end{document}

