\section{Materials and Methods}

Here we describe a dynamic model that represents dynamics of predation and migration considering biological and geographical conditions defined by ecological networks. The biological conditions are the trophic relationships among species living in a same region, and the geographical conditions is the connectivity among different regions. 

%This model can help to simulate how changes of environmental control mechanisms can affect those ecological relationships and to describe it we will use some descriptions as defined in Table 1.

We use a Monte Carlo (MC) approach to simulate the dynamics of predation and migration in an ecosystem for different timeline. Different patches are represented as nodes of a geographical neighborhood network and the connectivity of those patches are represented as edges of this network. At each patch $i$ there are $N_{i}^{k}$ individuals of each $k$ species. The trophic relationships among these species are represented by a directed network in which each node represents a species and each edge represents a trophic relationship between a pair of species in which the origin of the edge represents a prey species and its target represents a predator species. This network is called food-web and it is the basis for running the predation dynamic simulation.

\subsection{Predation dynamics}

At each time of the MC simulation we leave all the individuals of each patch $i$ to be choosen by a \emph{Multinomial Distribution} \cite{levin1981representation}. The chosen individual $k$ can have three different behaviors: 
\begin{enumerate}
\item it can die for natural reasons; 
\item it can eat one individual among its prey; 
\item if $k$ have eaten an individual among its prey, so it can give an offspring. 
\end{enumerate}

If $k$ individual is not a predator (if it is a basal species) the model assumes it has infinity food suply and the only possible behaviors are 1. and 2.. For each MC time-step $mc$, this simulation is repeated for all individuals of each patch of the landscape. The births will occur only if there is free space in the patch $i$, that means, if the number of individuals alive at $i$ is lower than its carrying capacity ($cc$). For each time $t$ in which one individual of species $k$ gives an offspring in a patch $i$, its number of individuals in this patch will be increased by 1; for each time $t$ in which one individual of species $k$ dies naturally or by predation in a patch $i$, its number of individuals will be decreased by 1. In Figure \ref{fig:Fluxogram} we show a fluxogram that summarizes the running of the predation dynamic of the model. 

\vspace{0.25cm}
\subsubsection{Predation Equations}

\vspace{0.25cm}
\begin{itemize}
\item \textbf{General Equation}
\end{itemize}

\begin{equation}
 \left\{\left[ 1 - NDp\left(k\right) \right]\left[ \sum_{b \in H\left(k\right)}\rho\left(b\right)Dp\left(b\right) \right]\left[Bp\left(k\right)\right] - \left[\sum_{c \in P\left(k\right)}\rho\left(c\right)\left( 1-NDp \left(c \right) \right)\frac{\rho\left(k\right)}{\sum_{d \in H\left(c\right)}\rho\left(d\right)}Dp\left(k\right)\right] - \left[NDp\left(k\right)\right] \right\} 
\end{equation}

%%%%%%%%%%%%%%%%%%
%%%%%%%%%%%%%%%%%%% B I R T H      P R O B A B I L I T Y
%%%%%%%%%%%%%%%%%%
\vspace{1cm}
\begin{itemize}
\item \textbf{Self-Organized Parameters: Birth Probability}
\end{itemize}

\begin{align*}
Bp\left(k\right) & = \tikz[baseline]{\node[fill=blue!20,ellipse,anchor=base] (t1) {$\left[1 - \rho\left(k\right)\right]$};} \times \tikz[baseline]{\node[fill=red!20,anchor=base] (t2) {$\left[\sum_{b \in H\left(k\right)}\rho\left(b\right)\left(1-\sum_{c \in P\left(b\right)}\rho\left(c\right)\right)\right]$};} \\
& \times \tikz[baseline]{\node[fill=green!20,anchor=base] (t3) {$\left[1-\sum_{c \in P\left(k\right)}\rho\left(c\right) \right]$};}
\end{align*}

\textbf{Where:} Availability of Resources of Basal Species is 1.0

%%%%%%%%%%%%%%%%%%%%%%%%%%
%%%%%%%%%%%%%%%%%%%%%%%%%%% D E A T H     P R O B A B I L I T Y
%%%%%%%%%%%%%%%%%%%%%%%%%%
\vspace{1cm}
\begin{itemize}
\item \textbf{Self-Organized Parameters: Death Probability}
\end{itemize}

\begin{align*}
Dp\left(k\right) & = \tikz[baseline]{\node[fill=blue!20,ellipse,anchor=base] (t1) {$\left[\rho\left(k\right)\right]$};} \times \tikz[baseline]{\node[fill=red!20,anchor=base] (t2) {$\left[\sum_{b \in H\left(k\right)}\left(1-\rho\left(b\right)\right)\left(\sum_{c \in P\left(b\right)}\rho\left(c\right)\right)\right]$};} \\ 
& \times \tikz[baseline]{\node[fill=green!20,anchor=base] (t3) {$\left[1-\sum_{c \in P\left(k\right)}\rho\left(c\right) \right]$};} 
\end{align*}

\textbf{Where:} Death Probability of Basal Species is 1.0

%%%%%%%%%%%%%%%%%%%%%%%%%%
%%%%%%%%%%%%%%%%%%%%%%%%%%% N A T U R A L     D E A T H     P R O B A B I L I T Y
%%%%%%%%%%%%%%%%%%%%%%%%%%
\vspace{1cm}
\begin{itemize}
\item \textbf{Self-Organized Parameters: Natural Death Probability}
\end{itemize}

\begin{align*}
NDp\left(k\right) & = \tikz[baseline]{\node[fill=blue!20,ellipse,anchor=base] (t1) {$\left[\rho\left(k\right)\right]$};} \times \tikz[baseline]{\node[fill=red!20,anchor=base] (t2) {$\left[\sum_{b \in H\left(k\right)}\left(1-\rho\left(b\right)\right)\left(\sum_{c \in P\left(b\right)}\rho\left(c\right)\right)\right]$};} \\ 
& \times \tikz[baseline]{\node[fill=green!20,anchor=base] (t3) {$\left[1-\sum_{c \in P\left(k\right)}\rho\left(c\right) \right]$};} 
\end{align*}

%%%%%%%%%%%%%%%%
%%%%%%%%%%%%%%%% S P E C I E S     C A R R Y I N G     C A P A C I T Y
%%%%%%%%%%%%%%%%
\vspace{1cm}
\begin{itemize}
\item \textbf{Self-Organized Parameters: Carrying Capacity (for each species in each site)}
\end{itemize}

\begin{align*}
CC\left(k\right) & =  \tikz[baseline]{\node[fill=red!20,ellipse,anchor=base] (t2) {$\left[\sum_{b \in H\left(k\right)}\frac{\left(\rho\left(b\right)\right)}{\left(\sum_{c \in P\left(b\right)}\rho\left(c\right)\right)}\right]$};}
\end{align*}

%%%%%%%%%%%%%%%%
%%%%%%%%%%%%%%%% M O B I L I T Y 
%%%%%%%%%%%%%%%% 

\subsection{Dispersal Migration}

The model allows the Dispersal Migration of individuals living in any patche $i$ to each patch in its neighborhood. The $i$-neighborhood is given by the topology of the neighborhood network, here we consider a neighborhood defined by a regular 2-dimensional toroidal lattice. Is assumed that, for each time step, each species in the landscape has a level of preference to each patch. The preference of a species $k$ for a patch $i$ in time $t$ ($Pref_{i}^k(t)$) is defined as the difference between the number of individuals of species that are prey and species that are predators of $k$ in that patch. The migration of a species $k$ from a patch $i$ to a neighborhood patch $j$ will occur only if  $Pref_{j}^k(t) > Pref_{i}^k(t)$. The number of individuals of species $k$ that move from $i$ to $j$ must respect the threshold imposed by the carrying capacity of the target patch ($cc_j^k(t)$) and is defined as seen below:

\vspace{0.25cm}
\subsubsection{Migration Equations}

%%%%%%\begin{frame}
%%%%%%\frametitle{Mobility Dynamics}
%%%%%%%\includegraphics[width=1.0\textwidth]{./mobility.eps}
%%%%%%\end{frame}
%%%%%%

\vspace{1cm}
\begin{itemize}
    \item \textbf{Mobility of species \emph{sp} from site \emph{i} to \emph{j}} 
\end{itemize}

$\Delta N_{sp}\left(i\right) = \sum_{j \in Neigh\left(i\right)} \left( \left( N_{sp}(j) \; M_{sp}(j,i) \; - \; \tikz[baseline]{\node[fill=blue!20,anchor=base] (t1) {$N_{sp}(i) $};} \tikz[baseline]{\node[fill=red!20,anchor=base] (t2) {$M_{sp}(i,j)$};} \right) \right)$


\vspace{0.25cm}
\small{$M(i,j) = \overbrace{ \left[ \lambda^{i} \frac{ \Delta_{ij}\,f\,\Theta \left(\Delta_{ij}\,f\,\right) }{ \sum_{k \in Neigh\left(i\right)}\Delta_{ik}\,f\,\Theta \left(\Delta_{ik}\,f\,\right) }\right]}^{Biotic} \overbrace{\left[w_{ij} \frac{ \Delta_{ij}\,f_{\eta}\,\Theta \left(\Delta_{ij}\,f_{\eta}\,\right) }{ \sum_{k \in Neigh\left(i\right)}\Delta_{ik}\,f_{\eta}\,\Theta \left(\Delta_{ik}\,f_{\eta}\,\right) }\right]}^{Abiotic}$}

\vspace{0.5cm}
$$
\Delta_{ij}\,f = f^{i,j} - f^{j,i}\left\{ \begin{array}{rl}
 f^{i,j} = \rho_{H}(j) + \rho_{P}(i) \\
 f^{j,i} = \rho_{H}(i) + \rho_{P}(j) \\
       \end{array} \right.
$$

\vspace{0.5cm}
\centering $\lambda^{i} = \frac{1}{2} \left( 1 - RE^{i} \right)$ \\

\vspace{0.25cm}
\centering $ RE^{i} = \frac{ New^{t} }{ N^{t} }$

\vspace{0.25cm}
$$
 \Delta_{ij}\,f_{\eta} = f_{\eta}^{j} - f_{\eta}^{i} \left\{ \begin{array}{rl}
 f_{\eta}^{i} = \eta_{sp}^{*} - \eta_{sp}^{i} \\
 f_{\eta}^{j} = \eta_{sp}^{*} - \eta_{sp}^{i} \\
       \end{array} \right.
$$

\vspace{0.25cm}
$$w_{ij} = Connectivity \: between \: sites \: i and \: j$$

%%%%%%%%%%%%%%%%%%%%%% 
%%%%%%%%%%%%%%%%%%%%%%

\vspace{1cm}
% Do NOT remove this, even if you are not including acknowledgments

